\section{Introduction}
Fields were introduced in classical physics to resolve "spooky action at a distance" -- the idea that one object can affect another object instantly, regardless of how far apart they are. In a local theory, like Maxwell's electrodynamics, an electron doesn't instantly feel the force from a distant proton. Instead, the proton modifies the electromagnetic field in its immediate vicinity, and changes in the field then propagate outward at the speed of light. The electron feels the force from the proton only when the changes in the field reach the electron. In more mathematically-precise language, a local theory is one wherein the equations governing an object's motion depend only on the fields evaluated at the particle's position, not on the position of any other object. For example, the force felt by the electron depends only on the electromagnetic field at the electron's position, not on the instantaneous position of the proton. \\

\noindent It's important to note that the fields aren't just a bookkeeping trick to hide non-locality. For a theory to be truly local, the fields themselves need to evolve locally -- in electrodynamics, this is ensured by Maxwell's equations. Changes in the proton's position affect the field, but that influence propagates through the field at a finite speed. The field at the electron's position at time $t$ depends on the proton's state only at an earlier time. By contrast, Coulomb's law describes an instantaneous interaction and is therefore non-local. A field defined by Coulomb's law alone simply hides this non-locality; it is only through the $\partial_t$ terms in Maxwell's equations that genuine locality and finite propagation speed are enforced. \\

\noindent From a philosophical perspective, it might seem natural to you that the force a particle feels should only depend on its immediate surroundings, not some arbitrary point some distance away. How would an electron "know" about a proton located arbitrarily far away? Einstein's special relativity reinforces this viewpoint: if causal influences \textit{could} propagate instantly, some reference frames would see the causal influence travel backwards in time, violating causality.\\

\noindent Over time, it became clear that fields are not merely devices used to enforce locality -- they are physical, dynamical entities. One major piece of evidence for this is that electromagnetic waves can propagate in vacuum, carrying energy and momentum\footnote[2]{Of course, this energy and momentum can only be measured once the waves interact with something. The term “existence” is inherently slippery: what does it really mean for a field to be a “real, physical object”? In practice, we define the existence of the field through its effects on other things. But the fact that these fields can propagate indefinitely, even in the absence of sources, gives good reason to treat them as dynamical entities in their own right. After all, how is a particle any more “real”? Both particles and fields are considered real only insofar as they influence other objects. If a field can exist independently and exert observable effects, is there any meaningful philosophical distinction between a field and a particle?}. During the quantum revolution, it was also discovered that particles exhibit both wave-like and particle-like behavior. If particles like electrons (once thought of as fundamental) and waves like light (originally viewed as disturbances in a field) are to be treated on equal footing, there are two philosophical approaches: 
\begin{itemize}
    \item Treat particles as fundamental, and view fields as effective descriptions of a large number of particles
    \item Treat the fields as fundamental, where wave behavior is natural, and interpret particles as quantized excitations of these fields
\end{itemize}
It turns out that the second option is far more powerful. This leads to quantum field theory -- a framework in which particles are manifestations of underlying quantum fields.